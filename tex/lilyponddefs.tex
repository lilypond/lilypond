%%% lilyponddefs.tex -- TeX macros for LilyPond output.
%%%
%%%  source file of the GNU LilyPond music typesetter
%%% 
%%% (c)  1998--2004 Jan Nieuwenhuizen <janneke@gnu.org>
%%%                 Han-Wen Nienhuys <hanwen@cs.uu.nl>
%%%                 Mats Bengtsson <mats.bengtsson@s3.kth.se>
%%%
%% Avoid \par while reading this file.
\edef\lilyponddefsELC{\the\endlinechar}%
\endlinechar -1\relax

%% This runs with plain TeX, LaTeX, pdftex, and texinfo.
%%
%% To avoid interferences, lilyponddefs.tex must be loaded within a group.
%% It is loaded only once, so the definitions must be global.
%%
%% The overall structure of a file created by LilyPond is as follows:
%%
%%   <lilypond parameter definitions>
%%   <font setup>
%%   \ifx\lilypondstart \undefined
%%     \input lilyponddefs
%%   \fi
%%   \lilypondstart
%%   <note output>
%%   \lilypondend

\newdimen\lytempdim
\newdimen\outputscale

%% Handy macros from the LaTeX manual.
\long\gdef\lilypondfirst#1#2{#1}
\long\gdef\lilypondsecond#1#2{#2}
\gdef\lilypondifundefined#1{
  \expandafter\ifx\csname#1\endcsname\relax
    \expandafter\lilypondfirst
  \else
    \expandafter\lilypondsecond
  \fi
}

%% Urgh.  Lilypond uses EC fonts, but texinfo is based on CM.  We thus
%% have to handle T1 font encoding by ourselves.  Note that the following
%% code only provides the texinfo interface, not complete access to all
%% EC glyphs.

\begingroup
\catcode `\@=11\relax
\gdef\lilypondECencoding{
  \def\"##1{
    {\accent4 ##1}}
  \def\'##1{
    {\accent1 ##1}}
  \def\,##1{
    {\leavevmode
     \setbox\z@\hbox{##1}
     \ifdim\ht\z@=1ex
       \accent11 ##1
     \else
       {\ooalign{
          \unhbox\z@
          \crcr
          \hidewidth
          \char11
          \hidewidth}}
     \fi}}
  \def\=##1{
    {\accent9 ##1}}
  \def\^##1{
    {\accent2 ##1}}
  \def\`##1{
    {\accent0 ##1}}
  \def\~##1{
    {\accent3 ##1}}
  \def\dotaccent##1{
    {\accent10 ##1}}
  \def\H##1{
    {\accent5 ##1}}
  \def\ringaccent##1{
    {\accent6 ##1}}
% \def\tieaccent##1{}        % unsupported: this is TS1
  \def\u##1{
    {\accent8 ##1}}
  \def\ubaraccent##1{
    {\o@lign{
       \relax
       ##1
       \crcr
       \hidewidth
       \sh@ft{29}\vbox to.2ex{
         \hbox{\char9}
         \vss}
       \hidewidth}}}
  \def\udotaccent##1{
    {\o@lign{
       \relax
       ##1
       \crcr
       \hidewidth
       \sh@ft{10}.
       \hidewidth}}}
  \def\v##1{
    {\accent7 ##1}}

  \chardef\exclamdown=189
  \chardef\questiondown=190

  \def\aa{
    \ringaccent{a}}
  \def\AA{
    \ringaccent{A}}
  \chardef\AE=198
  \chardef\ae=230
  \chardef\ptexi=25
  \chardef\j=26
  \chardef\L=138
  \chardef\l=170
  \chardef\O=216
  \chardef\o=248
  \chardef\OE=215
  \chardef\oe=247
  \chardef\ss=255
}
\endgroup

\gdef\lilypondstart{
  \frenchspacing
  \begingroup
  \catcode `\@=11\relax
  %% \@nodocument is defined as \relax after `\begin{document}'
  \lilypondifundefined{@nodocument}
    {%% either plain TeX or texinfo or not at the beginning of LaTeX input
     \def\x{
       \endgroup
       \def\lilypondfontencoding####1{
         \lilypondECencoding}}}
    {%% FIXME: a4
     %% provide a proper LaTeX preamble (for A4 paper format)
     \def\x{
       \endgroup
       \def\lilyponddocument{}
       \def\lilypondfontencoding####1{
         \fontencoding{####1}
         \selectfont}
       \documentclass[a4paper]{article}
       %% safe-mode
       \nofiles
       \usepackage[\lilypondpaperinputencoding]{inputenc}
       \pagestyle{empty}
       \lilypondifundefined{lilypondclassic}
         {%% Nullify [La]TeX page layout settings, page layout by LilyPond.
          \topmargin-1in
          \headheight0pt\headsep0pt
          \oddsidemargin-1in
          \evensidemargin\oddsidemargin}
         {%% Center staves horizontally on page
          \ifdim\lilypondpaperlinewidth\lilypondpaperunit > 0pt
            \hsize\lilypondpaperlinewidth\lilypondpaperunit
            \lytempdim \paperwidth
            \advance\lytempdim -\the\hsize
            \lytempdim 0.5\lytempdim
            \advance\lytempdim -1in
            \oddsidemargin \lytempdim
            \evensidemargin \lytempdim
          \fi}
       \parindent 0pt
       %% TEXINFO workaround: \begin is defined as \outer, use \csname.
       \csname begin\endcsname{document}}}
  \x}

\gdef\lilypondend{
  \lilypondifundefined{lilypondbook}
    {\lilypondifundefined{lilypondpaperlastpagefill}
       {\vskip 0pt plus\lilypondpaperinterscorelinefill00 fill}
       {}}
    {}
  \begingroup
  \lilypondifundefined{lilyponddocument}
    {\def\x{\endgroup}}
    {\def\x{\endgroup\csname end\endcsname{document}}}
  \x}

%% Allow overriding of pagebreak
\lilypondifundefined{lilypondpagebreak}
  {\lilypondifundefined{@nodocument}
     {\gdef\lilypondpagebreak{\eject}}
     {\gdef\lilypondpagebreak{\newpage}}}
  {}
      
%% Include \special only once.
\gdef\lilypondspecial{
  \special{header=music-drawing-routines.ps}
  \gdef\lilypondspecial{}}

%% The feta characters.
\input feta20

\global\font\fetasixteen = feta16
\gdef\fetafont{\fetasixteen}
\gdef\fetachar#1{\hbox{\fetasixteen#1}}

\gdef\topalign#1{\vbox to 0pt{\hbox{#1}\vss}}
\gdef\leftalign#1{\hbox to 0pt{#1\hss}}

\gdef\lyitem#1#2#3{
  \topalign{\raise#2\outputscale\leftalign{\kern#1\outputscale#3}}}

\gdef\lybox#1#2#3#4#5{
  \lytempdim\baselineskip
  \advance\lytempdim-#4\outputscale
  \raise\lytempdim
  \vbox to#4\outputscale{
    \leftalign{\kern#1\outputscale\lower#2\outputscale\topalign{#5}}
    \vss}}

\gdef\lyvrule#1#2#3#4{
  \kern#1\outputscale
  \vrule width #2\outputscale depth #3\outputscale height #4\outputscale}

%% FIXME: 'interscoreline' and 'lilypondPAPERinterscoreline
\lilypondifundefined{lilypondpaperinterscorelinefill}
  {\gdef\lilypondpaperinterscorelinefill{0}}
  {\gdef\lilypondpaperinterscorelinefill{1}}

%% Allow overriding of interscoreline, e.g. for lilypond.py's --preview
\lilypondifundefined{interscoreline}
  {\lilypondifundefined{lilypondclassic}
     {\gdef\interscoreline{}}
     {\gdef\interscoreline{
        \vskip\lilypondpaperinterscoreline\lilypondpaperunit
        plus \lilypondpaperinterscorelinefill fill}}}
  {}

%% Include postscript definitions unless using PDFTeX,
%% in that case use pdf definitions.
%% MiKTeX workaround: use \csname.
\lilypondifundefined{lilypondpostscript}
  {\lilypondifundefined{pdfoutput}
     {\input lily-ps-defs }
     {\pdfoutput = 1
      \input lily-pdf-defs }}
  {}

%% Restore newline functionality (disabled to avoid \par).
\endlinechar \lilyponddefsELC
\endinput

%% end lilyponddefs.tex
